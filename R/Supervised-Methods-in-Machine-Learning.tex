% Options for packages loaded elsewhere
\PassOptionsToPackage{unicode}{hyperref}
\PassOptionsToPackage{hyphens}{url}
%
\documentclass[
]{article}
\usepackage{amsmath,amssymb}
\usepackage{lmodern}
\usepackage{iftex}
\ifPDFTeX
  \usepackage[T1]{fontenc}
  \usepackage[utf8]{inputenc}
  \usepackage{textcomp} % provide euro and other symbols
\else % if luatex or xetex
  \usepackage{unicode-math}
  \defaultfontfeatures{Scale=MatchLowercase}
  \defaultfontfeatures[\rmfamily]{Ligatures=TeX,Scale=1}
\fi
% Use upquote if available, for straight quotes in verbatim environments
\IfFileExists{upquote.sty}{\usepackage{upquote}}{}
\IfFileExists{microtype.sty}{% use microtype if available
  \usepackage[]{microtype}
  \UseMicrotypeSet[protrusion]{basicmath} % disable protrusion for tt fonts
}{}
\makeatletter
\@ifundefined{KOMAClassName}{% if non-KOMA class
  \IfFileExists{parskip.sty}{%
    \usepackage{parskip}
  }{% else
    \setlength{\parindent}{0pt}
    \setlength{\parskip}{6pt plus 2pt minus 1pt}}
}{% if KOMA class
  \KOMAoptions{parskip=half}}
\makeatother
\usepackage{xcolor}
\usepackage[margin=1in]{geometry}
\usepackage{color}
\usepackage{fancyvrb}
\newcommand{\VerbBar}{|}
\newcommand{\VERB}{\Verb[commandchars=\\\{\}]}
\DefineVerbatimEnvironment{Highlighting}{Verbatim}{commandchars=\\\{\}}
% Add ',fontsize=\small' for more characters per line
\usepackage{framed}
\definecolor{shadecolor}{RGB}{248,248,248}
\newenvironment{Shaded}{\begin{snugshade}}{\end{snugshade}}
\newcommand{\AlertTok}[1]{\textcolor[rgb]{0.94,0.16,0.16}{#1}}
\newcommand{\AnnotationTok}[1]{\textcolor[rgb]{0.56,0.35,0.01}{\textbf{\textit{#1}}}}
\newcommand{\AttributeTok}[1]{\textcolor[rgb]{0.77,0.63,0.00}{#1}}
\newcommand{\BaseNTok}[1]{\textcolor[rgb]{0.00,0.00,0.81}{#1}}
\newcommand{\BuiltInTok}[1]{#1}
\newcommand{\CharTok}[1]{\textcolor[rgb]{0.31,0.60,0.02}{#1}}
\newcommand{\CommentTok}[1]{\textcolor[rgb]{0.56,0.35,0.01}{\textit{#1}}}
\newcommand{\CommentVarTok}[1]{\textcolor[rgb]{0.56,0.35,0.01}{\textbf{\textit{#1}}}}
\newcommand{\ConstantTok}[1]{\textcolor[rgb]{0.00,0.00,0.00}{#1}}
\newcommand{\ControlFlowTok}[1]{\textcolor[rgb]{0.13,0.29,0.53}{\textbf{#1}}}
\newcommand{\DataTypeTok}[1]{\textcolor[rgb]{0.13,0.29,0.53}{#1}}
\newcommand{\DecValTok}[1]{\textcolor[rgb]{0.00,0.00,0.81}{#1}}
\newcommand{\DocumentationTok}[1]{\textcolor[rgb]{0.56,0.35,0.01}{\textbf{\textit{#1}}}}
\newcommand{\ErrorTok}[1]{\textcolor[rgb]{0.64,0.00,0.00}{\textbf{#1}}}
\newcommand{\ExtensionTok}[1]{#1}
\newcommand{\FloatTok}[1]{\textcolor[rgb]{0.00,0.00,0.81}{#1}}
\newcommand{\FunctionTok}[1]{\textcolor[rgb]{0.00,0.00,0.00}{#1}}
\newcommand{\ImportTok}[1]{#1}
\newcommand{\InformationTok}[1]{\textcolor[rgb]{0.56,0.35,0.01}{\textbf{\textit{#1}}}}
\newcommand{\KeywordTok}[1]{\textcolor[rgb]{0.13,0.29,0.53}{\textbf{#1}}}
\newcommand{\NormalTok}[1]{#1}
\newcommand{\OperatorTok}[1]{\textcolor[rgb]{0.81,0.36,0.00}{\textbf{#1}}}
\newcommand{\OtherTok}[1]{\textcolor[rgb]{0.56,0.35,0.01}{#1}}
\newcommand{\PreprocessorTok}[1]{\textcolor[rgb]{0.56,0.35,0.01}{\textit{#1}}}
\newcommand{\RegionMarkerTok}[1]{#1}
\newcommand{\SpecialCharTok}[1]{\textcolor[rgb]{0.00,0.00,0.00}{#1}}
\newcommand{\SpecialStringTok}[1]{\textcolor[rgb]{0.31,0.60,0.02}{#1}}
\newcommand{\StringTok}[1]{\textcolor[rgb]{0.31,0.60,0.02}{#1}}
\newcommand{\VariableTok}[1]{\textcolor[rgb]{0.00,0.00,0.00}{#1}}
\newcommand{\VerbatimStringTok}[1]{\textcolor[rgb]{0.31,0.60,0.02}{#1}}
\newcommand{\WarningTok}[1]{\textcolor[rgb]{0.56,0.35,0.01}{\textbf{\textit{#1}}}}
\usepackage{graphicx}
\makeatletter
\def\maxwidth{\ifdim\Gin@nat@width>\linewidth\linewidth\else\Gin@nat@width\fi}
\def\maxheight{\ifdim\Gin@nat@height>\textheight\textheight\else\Gin@nat@height\fi}
\makeatother
% Scale images if necessary, so that they will not overflow the page
% margins by default, and it is still possible to overwrite the defaults
% using explicit options in \includegraphics[width, height, ...]{}
\setkeys{Gin}{width=\maxwidth,height=\maxheight,keepaspectratio}
% Set default figure placement to htbp
\makeatletter
\def\fps@figure{htbp}
\makeatother
\setlength{\emergencystretch}{3em} % prevent overfull lines
\providecommand{\tightlist}{%
  \setlength{\itemsep}{0pt}\setlength{\parskip}{0pt}}
\setcounter{secnumdepth}{-\maxdimen} % remove section numbering
\ifLuaTeX
  \usepackage{selnolig}  % disable illegal ligatures
\fi
\IfFileExists{bookmark.sty}{\usepackage{bookmark}}{\usepackage{hyperref}}
\IfFileExists{xurl.sty}{\usepackage{xurl}}{} % add URL line breaks if available
\urlstyle{same} % disable monospaced font for URLs
\hypersetup{
  pdftitle={UntitledAssignment 3, ``Supervised learning final project''},
  pdfauthor={Alejandro Pachón, Santiago Meza, Alexander Morgan},
  hidelinks,
  pdfcreator={LaTeX via pandoc}}

\title{UntitledAssignment 3, ``Supervised learning final project''}
\author{Alejandro Pachón, Santiago Meza, Alexander Morgan}
\date{2023-05-29}

\begin{document}
\maketitle

\hypertarget{sensores}{%
\subsection{Sensores}\label{sensores}}

\hypertarget{mq2}{%
\subsubsection{MQ2:}\label{mq2}}

El funcionamiento del sensor MQ2 se basa principalmente en medir la
variación en la conductividad de la alterada por la presencia de gases
en el aire, estas variaciones producen señales eléctricas que se
interpretan de diferentes maneras para llegar al valor de la
concentración de gases en el ambiente. Para información más detallada
consultar el
\href{chrome-extension://efaidnbmnnnibpcajpcglclefindmkaj/https://www.pololu.com/file/0J309/MQ2.pdf}{datasheet}.

Para la recolección de datos usando el sensor MQ2, se simuló el ambiente
de una cueva volcánica, la cual constaba con irregularidades en el suelo
como si fueran grietas, en estas también se agrego alcohol atomizado, se
humedecieron partes del ambiente con gasolina y se agrego un poco de gas
de un encendedor para simular los gases tóxicos que dichas cuevas pueden
llegar a expulsar.

\hypertarget{dht11}{%
\subsubsection{DHT11:}\label{dht11}}

El sensor DHT11 es un sensor capacitivo que mide la humedad y
temperatura relativa del aire, incluye también un termistor interno el
cual mide la temperatura del ambiente, y muestra los datos mediante una
señal digital en el pin de datos, Para información más detallada revisar
el
\href{chrome-extension://efaidnbmnnnibpcajpcglclefindmkaj/https://www.mouser.com/datasheet/2/758/DHT11-Technical-Data-Sheet-Translated-Version-1143054.pdf}{datasheet}.

\#\#imagen del sensor\#\#\includegraphics{}

En cuanto a la recolección de datos, se tuvieron en cuenta dos de los
ambientes simulados, la jungla y el desierto, ya que estos ambientes
cambian abruptamente las magnitudes de temperatura y calor debido a la
relación entre estas, ya que en el ambiente de jungla se buscaba
encontrar una alta humedad y mantener una temperatura relativa estable
para que le sensor midiera la relación normalizada en este ambiente, en
el ambiente de desierto se busco lo contrario, y se evidencio en el
dataset como al elevar la temperatura el factor de humedad disminuye
considerable y constantemente.

\hypertarget{hc-sr04}{%
\subsubsection{Hc Sr04}\label{hc-sr04}}

El funcionamiento básico del sensor Hc Sr04 (sensor de ultrasonido) es
de la emisión y recepción de pulsos ultrasónicos por sus transductores y
medir el tiempo que este tarda en llegar al receptor del sensor. Para
información más detallada puede consultar el
\href{chrome-extension://efaidnbmnnnibpcajpcglclefindmkaj/https://leantec.es/wp-content/uploads/2019/06/Leantec.ES-HC-SR04.pdf}{datasheet}.

\#\#imagen del sensor\#\#

El sensor de ultrasonido resultó útil en los 3 ambientes, ya que cada
uno de estos simulo una distancia de espacio diferente, en el caso de la
cueva, al ser ambiente con menor espacio, comparando sus medidas con los
demás ambientes son considerables, en el ambiente de jungla al tener
diferentes obstáculos y al estar en diferentes alturas, se buscaba
simular un espacio con árboles, hojas y/o arbustos, los cuales
dificultaran la medición de un espacio exterior, y finalmente en el
espacio de desierto se aplicó lo contrario a los dos anteriores
escenarios, ya que als er un desierto el espacio de este es muy amplio y
esto se refleja en las medidas del dataset.

\begin{Shaded}
\begin{Highlighting}[]
\FunctionTok{summary}\NormalTok{(cars)}
\end{Highlighting}
\end{Shaded}

\begin{verbatim}
##      speed           dist       
##  Min.   : 4.0   Min.   :  2.00  
##  1st Qu.:12.0   1st Qu.: 26.00  
##  Median :15.0   Median : 36.00  
##  Mean   :15.4   Mean   : 42.98  
##  3rd Qu.:19.0   3rd Qu.: 56.00  
##  Max.   :25.0   Max.   :120.00
\end{verbatim}

\hypertarget{including-plots}{%
\subsection{Including Plots}\label{including-plots}}

You can also embed plots, for example:

\includegraphics{Supervised-Methods-in-Machine-Learning_files/figure-latex/pressure-1.pdf}

Note that the \texttt{echo\ =\ FALSE} parameter was added to the code
chunk to prevent printing of the R code that generated the plot.

\end{document}
